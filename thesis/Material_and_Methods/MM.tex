%=== Materials and Methods ===
\chapter{2. Material and Methods}
\section{Data Collection}

Lorem ipsum dolor sit amet, consectetur adipiscing elit. Sed sit amet lacinia urna, eu posuere enim. Donec vulputate finibus molestie. Praesent sit amet lacus orci. Etiam molestie augue a ligula ullamcorper, lacinia lobortis elit semper in Fig \ref{fig:singmap}. %reference to a figure.

%Adding a figure
\begin{figure}[!htb]
    \centering
    \includegraphics[width=\textwidth]{fig/SingaporeMap.png}/
    \caption[Singapore Map] \\{\footnotesize Adapted from \cite{sapp_2004}}
    \label{fig:singmap} %use label for figures and tables
\end{figure}

\section{Maecenas lobortis}

\nomenclature{PCM}{Pulse code modulation} % DEFINE ACRONYM/SYMBOLS USING THIS FUNCTION

The summary of speaker information for the evaluation sources and analyses of \textbf{pulse code modulation (PCM)} used in this report is presented in Table \ref{tab:sourceinfo}. % REF TO TABLES IN APPENDIX. 


\subsection{Frequency-domain ipsum dolor}
The various types of window functions and their properties are reviewed in Harrris et al.\parencite*{harris1978use}. For this project, we use the periodic Hamming window defined by

%ADDING EQUATION
\begin{equation}
    w(\tau) = \begin{cases}
    0.54 - 0.46\cos\left(\dfrac{2\pi\tau}{R}\right)& 0 \le \tau < R,\\
    0, & \text{otherwise,}
    \end{cases}
\end{equation}
with $75\%$ overlap.

\section{Vestibulum tempus}

A set of $N$ source signals $\bm{s}(\tau) = [s_0(\tau), \dots, s_{N-1}(\tau)]^T \in \R^{N \times T_t}$ over the samples $0\le \tau < T_t$, such that $s_n = [s_n(\tau = 0), s_n(1), \dots, s_n(T_t - 1)]$ for $0 \le n < N$, undergoes mixing whose result is observed by an array of $M$ sensors (in this project's context, microphones). The observed signals are represented by $\bm{x}(\tau) = [x_0(\tau), \dots, x_{M-1}(\tau)]^T$, such that $x_m = [x_m(\tau = 0), x_m(1), \dots, x_m(T_t - 1)] \in \R^{M \times T}$ for $0 \le m < M$. The sources are mixed convolutively such that
\begin{equation}
    \bm{x}(\tau) = \sum_{k = 0}^{K - 1} \bm{A}_k \bm{s}(\tau - k)
\end{equation}
where $\bm{A}_k \in \R^{M\times N}$ with $K$ often assumed, in practice, to be some finite positive integer. This means that $\bm{x}(\tau)$ can be thought of as the output signal of a finite impulse response filter with input $\bm{s}(\tau)$ and $\bm{A}_k$ representing the coefficients of the $k$th-order filter. 

However, solving a convolutive system in the time domain is difficult and requires significant computational power. To simplify the problem, BSS problems are often transformed into the frequency domain such that
\begin{equation}
    \bm{X}(\omega, t) = \bm{A}(\omega) \bm{S}(\omega, t)
\end{equation}
where $\bm{X}(\omega, t) \in \C^{M \times T_f}$ and $\bm{S}(\omega, t) \in \C^{N \times T_f}$ are the frequency-domain representation of the observed signals and the source signals respectively \parencite{smaragdis1998blind}.

The configuration of 2 sources used is detailed in Figures \ref{fig:2mix1} to \ref{fig:3mix2}. %ADD REF TO APPENDIX FIGURES

%=== END OF MM ===
\newpage
